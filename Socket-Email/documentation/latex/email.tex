\section{Gửi và nhận email}
\subsection{Nhận mail}
\begin{itemize}
\item Giao thức sử dụng: POP3.
\item Nhà cung cấp mail: Microsoft Outlook.
\item Sử dụng đa luồng để tối ưu hóa tác vụ.
\end{itemize}
Các hàm sử dụng cho quá trình nhận mail (POP3$\_$service.py):
\begin{lstlisting}
def get_mails():
\end{lstlisting}
Chức năng: Lấy toàn bộ mail đang có trong mail box, tách lấy \textit{email người gửi}, \textit{tiêu đề} và \textit{nội dung} (nếu có), đưa vào hàng đợi để luồng khác xử lý, sau đó thì xóa mail để đảm bảo không trùng lắp với thư cũ.\\
Quá trình thực hiện:
\begin{itemize}
 \item Tạo kết nối POP3 tới server (có sử dụng SSL).
 \item Gửi username (mail) và password để xác thực.
 \item Lấy số lượng mail hiện có qua lệnh \lstinline{LIST}.
 \item Lấy thông tin từng thư qua lệnh \lstinline{RETR}.
 \item Tách lấy thông tin.
 \item Xóa thư với lệnh \lstinline{DELE}.
 \item Gửi lệnh \lstinline{QUIT} ngắt kết nối.
\end{itemize} 

\begin{lstlisting}
def loop():
\end{lstlisting}
Chức năng: Do tính chất của POP3 không cập nhật thư mới, cần phải reload lại sau một khoảng thời gian, nên hàm này sẽ thực hiện việc lấy thư liên tục (gọi tới hàm \lstinline{get_mails()}) khi chương trình khởi chạy.

\subsection{Gửi mail}
\begin{itemize}
\item Giao thức sử dụng: SMTP.
\item Nhà cung cấp mail: Microsoft Outlook.
\item Sử dụng đa luồng để tối ưu hóa tác vụ.
\end{itemize}
Các hàm sử dụng cho quá trình gửi mail (SMTP$\_$service.py):
\begin{lstlisting}
def send(to_: str, subject_: str, content_, file_name):
\end{lstlisting}
Chức năng: Gửi thư đến địa chỉ \lstinline{to_}, với tiêu đề \lstinline{subject_}, nội dung \lstinline{content_} và tên file \lstinline{file_name} nếu nội dung cần gửi là file (file ảnh, video...)\\
Tham số: 
\begin{itemize}
\item \lstinline{ to_}: kiểu \lstinline{str}, là địa chỉ mail người nhận (ví dụ \lstinline{abc@gmail.com}).
\item \lstinline{subject_}: kiểu \lstinline{str}, là tiêu đề của mail.
\item \lstinline{content_}: kiểu \lstinline{str} hoặc \lstinline{bytes}. Nếu là \lstinline{str}, thực hiện gửi như văn bản bình thường. Nếu là  \lstinline{bytes}, gửi như một file với tên file là \lstinline{file_name}.
\item \lstinline{file_name}: kiểu \lstinline{str}, tên file (mặc định là "x.txt").
\end{itemize}
Quá trình thực hiện:
\begin{itemize}
\item Tạo kết nối SMTP tới server (có sử dụng TLS) .
\item Gửi username (mail) và password để xác thực.
\item Soạn thư với định dạng \href{https://datatracker.ietf.org/doc/html/rfc1341}{MIME multipart}.
\item Nếu thư kiểu text, attach \lstinline{MIME text}.
\item Nếu thư kiểu bytes, attach \lstinline{MIME application} với tên file.
\item Gửi payload đến server và đóng kết nối.
\end{itemize}

\begin{lstlisting}
def safe_send(to_:str, subject_:str, content_, file_name):
\end{lstlisting}
Chức năng: Là wrapper cho hàm \lstinline{send}, thực hiện xử lý ngoại lệ: thực hiện gửi thư lại cho đến khi thành công (do trong một số trường hợp, gửi mail quá nhanh server sẽ từ chối nên phải thực hiện lại).\\
Tham số: tương tự \lstinline{send}.

\begin{lstlisting}
def send_threading(to_:str, subject_:str, content_, file_name = "x.txt"):
\end{lstlisting}
Chức năng: tạo threading để gửi thư, các tham số sẽ truyền trực tiếp vào cho hàm \lstinline{safe_send}. (Đây là hàm sẽ được gọi để gửi thư khi được import)\\
