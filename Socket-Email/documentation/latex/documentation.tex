\section{Sưu liệu cho các hàm}
\subsection{Phân tích câu lệnh}
Hàm phân tích câu lệnh được định nghĩa là:
\begin{lstlisting}
def command_parser(message, sender_email_address):
\end{lstlisting}
Chức năng: phân tích câu lệnh được gửi đến, gọi hàm tương ứng với yêu cầu của câu lệnh.\\
Tham số:
\begin{itemize}
\item \lstinline{message}: câu lệnh được gửi đến
\item \lstinline{sender_email_address}: địa chỉ email của người gửi.
\end{itemize}
Không có giá trị trả về.\\

Hàm \lstinline{get_corresponding_ip}
\begin{lstlisting}
def get_corresponding_ip(sender_email_address):
\end{lstlisting}
Chưc năng: lấy ra địa chỉ IP tương ứng với \lstinline{sender_email_address}.\\
Tham số:
\begin{itemize}
\item \lstinline{sender_email_address}: địa chỉ email của người gửi.
\end{itemize}
Trả về địa chỉ IP tương ứng, trả về None nếu không tồn tại.
\subsection{Module handle}
Các hàm được định nghĩa trong module handle.py được sử dụng để xử lý các yêu cầu từ người dùng.
\begin{lstlisting}
def authorize(email, ip):
\end{lstlisting}
Chức năng: authorize địa chỉ email với IP tương ứng.\\
Thám số: 
\begin{itemize}
\item \lstinline{email}: địa chỉ email.
\item \lstinline{ip}: địa chỉ IP.
\end{itemize}
Trả về True nếu thành công, False nếu không thành công.

\begin{lstlisting}
def list_ip():
\end{lstlisting}
Chức năng: trả về danh sách các địa chỉ IP đang được kết nối.

\begin{lstlisting}
def disconnect(email):
\end{lstlisting}
Chức năng: disconnect địa chỉ email hiện tại với IP tương ứng.
Tham số:
\begin{itemize}
\item \lstinline{email}: địa chỉ IP tương ứng.
\end{itemize}
Không có giá trị trả về.

\begin{lstlisting}
def find_corresponding_email(ip_address):
\end{lstlisting}
Chức năng: tìm địa chỉ email tương ứng với IP được truyền vào.
Tham số:
\begin{itemize}
\item \lstinline{ip_address}: địa chỉ IP.
\end{itemize}
Trả về: địa chỉ email tương ứng đang điều khiển địa chỉ IP này.

\begin{lstlisting}
def delete_ip_from_list(ip_address):
\end{lstlisting}
Chức năng: xóa địa chỉ IP ra khỏi danh sách IP kết nối.
Tham số: 
\begin{itemize}
\item \lstinline{ip_address}: địa chỉ IP.
\end{itemize}
Không có giá trị trả về.

\begin{lstlisting}
def remove_this_connection(conn, ip_address):
\end{lstlisting}
Chức năng: loại bỏ kết nối tương ứng với IP.
Tham số:
\begin{itemize}
\item \lstinline{conn}: socket tương ứng.
\item \lstinline{ip_address}: địa chỉ IP.
\end{itemize}
Không có giá trị trả về.

\begin{lstlisting}
def __init__():
\end{lstlisting}
Chức năng: khởi tạo những giá trị liên quan.
\subsection{Process/Application}
Server sẽ gửi lệnh \textbf{APP$\_$PRO} đến client, sau đó sẽ gửi số tương ứng với lệnh.
\begin{itemize}
\item Gửi số 1 nếu ta cần đưa ra danh sách process/application. Sau đó gửi message là \textbf{APPLICATION} hoặc \textbf{PROCESS} tương ứng với yêu cầu của lệnh đến client. Client nhận thông điệp rồi gửi lại danh sách application/process dưới dạng một data frame.
\item Gửi số 0 nếu ta cần kill một process/application. Sau đó ta sẽ gửi ID/PID của đối tượng cần kill. Client nhận thông điệp rồi gửi thông tin về quá trình kill lại cho server.
\end{itemize}
Các hàm cho quá trình này như sau:\\
\begin{itemize}
\item Ở server (mail$\_$handler.py, app$\_$process$\_$server.py):
\begin{lstlisting}
def list_process(ip_address):
\end{lstlisting}
Chức năng: gửi yêu cầu đến client, nhận kết quả rồi gửi mail lại cho người dùng.\\
Tham số: 
\begin{itemize}
\item \lstinline{ip_address}: kiểu tuple(str, int) chứa địa chỉ ip và port của client gửi tới.
\end{itemize}
Trả về: Không có giá trị trả về.
\begin{lstlisting}
def _list(conn:socket.socket, s):
\end{lstlisting}
Chức năng: gửi yêu cầu danh sách tiến trình/ứng dụng đến client, nhận kết quả từ client.\\
Tham số: 
\begin{itemize}
\item \lstinline{conn}: socket kết nối server với client.
\item \lstinline{s}: chuỗi, có giá trị là \textbf{PROCESS} hoặc \textbf{APPLICATION}, thể hiện yêu cầu gửi danh sách tiến trình hoặc ứng dụng.
\end{itemize}
Trả về: Danh sách tiến trình/ứng dụng ở dạng chuỗi.
\begin{lstlisting}

def send_kill(conn:socket.socket, process_id): 
\end{lstlisting}
Chức năng: gửi yêu cầu tắt một tiến trình.\\
Tham số: 
\begin{itemize}
\item \lstinline{conn}: socket kết nối server với client.
\item \lstinline{process_id}: id của tiến trình cần tắt.
\end{itemize}
Trả về: kết quả kill thành công/không thành công.\\
Thư viện sử dụng: pickle, struct, pandas.
\item Ở client (app$\_$process$\_$client.py):
\begin{lstlisting}
def app_process(client):
\end{lstlisting}
Chức năng: nhận yêu cầu từ server, xử lý yêu cầu rồi gửi lại kết quả tương ứng. \\
Tham số: 
\begin{itemize}
\item \lstinline{client}: socket kết nối server với client.
\end{itemize}
Trả về: Không có giá trị trả về.\\
Thư viện sử dụng: pickle, psutil, struct, os, subprocess.\\
Bên cạnh đó, server còn có các hàm (được cung cấp sẵn) phụ trách việc nhận dữ liệu:
\begin{lstlisting}
def recvall(sock, size):
def receive(client):
\end{lstlisting}
Ngoài ra, client còn có các hàm (được cung cấp sẵn) phục vụ việc gửi dữ liệu cho server và xử lý các yêu cầu về application/process trên máy.
\begin{lstlisting}
def send_data(client, data):
def list_apps():
def list_processes():
def kill(pid):
\end{lstlisting}
\end{itemize}






\subsection{Chụp màn hình}
Server sẽ gửi lệnh \textbf{LIVESCREEN} đến client, sau đó sẽ gửi lệnh tương ứng với thời gian cần chụp màn hình, thời gian mặc định là $0.5$ giây. Thời gian chụp không nên quá nhiều, vì giới hạn dung lượng tệp đính kèm trong email. Sau đó, server sẽ nhận ảnh gửi từ client, tạo thành một video.\\
Các hàm cho quá trình này như sau:\\
Ở server:
\begin{lstlisting}
def capture_screen(ip_address, time=0.5): # module handle
\end{lstlisting}
Chức năng: gửi yêu cầu chụp màn hình đến cho client, nhận hình ảnh từ client gửi lại, tạo video.\\
Ở client:
\begin{lstlisting}
def capture_screen(client):
\end{lstlisting}
Chức năng: nhận yêu cầu chụp màn hình từ server, chụp màn hình liên tục rồi gửi từng ảnh lại cho server.\\
Thư viện sử dụng: ImageGrab, io, time.\\
Ngoài ra, ở server còn có hàm sau dùng để tạo video từ những ảnh chụp màn hình nhận được từ client, sử dụng thư viện os, cv2.
\begin{lstlisting}
def create_video(image_folder: str):
\end{lstlisting}
\subsection{Webcam}
Server sẽ gửi lệnh \textbf{WEBCAM} đến client, sau đó sẽ gửi lệnh tương ứng với thời gian cần chụp màn hình, thời gian mặc định là $5$ giây. Client nhận thông điệp từ server, quay màn hình webcam trong khoảng thời gian đó, rồi gửi lại video cho server.\\
Các hàm cho quá trình này như sau:\\
Ở server:
\begin{lstlisting}
def capture_webcam(ip_address, time=5):
\end{lstlisting}
Chức năng: gửi yêu cầu ghi lại webcam đến cho client, nhận video từ client gửi về.\\
Ở client:
\begin{lstlisting}
def run(conn: socket.socket):
\end{lstlisting}
Chức năng: nhận yêu cầu từ server, quay màn hình webcam, gửi lại file video cho server.\\
Thư viện sử dụng: tempfile, cv2, time.\\
Ngoài ra, ở client còn có hàm sau dùng để gửi file về cho server:
\begin{lstlisting}
def send_file(conn: socket.socket):
\end{lstlisting}
\subsection{Keylogger}
Server sẽ gửi lệnh \textbf{KEYLOG} cho client. Do thiết kế của các hàm có sẵn cho việc keylog ở client, server sẽ gửi 2 lệnh \textbf{HOOK} đến client để lấy các hành động giữa 2 lần. Sau đó, một lệnh \textbf{PRINT} được gửi đến client để client gửi dữ liệu về các phím nhấn đã bắt được cho server. Server nhận kết quả từ client.\\
Các hàm cho quá trình này như sau:\\
- Ở server:
\begin{lstlisting}
def keylog(ip_address, time=10):
\end{lstlisting}
Chức năng: gửi yêu cầu keylog đến client, nhận kết quả từ client gửi về.\\
Sử dụng hàm \lstinline{sleep} của thư viện time.\\
- Ở client:
\begin{lstlisting}
def keylog(client):
\end{lstlisting}
Chức năng: nhận lệnh từ server, bắt các phím nhấn rồi gửi kết quả lại cho server, ngoài ra còn có chức năng phụ để có thể khóa phím người dùng (Không được đề cập trong chương trình).\\
Thư viện sử dụng: threading, keyboard, pynput.keyboard.\\
Ngoài ra, ở client còn có các hàm để bổ trợ cho hàm \lstinline{keylog} trong việc bắt phím nhấn cũng như gửi kết quả lại cho server.
\begin{lstlisting}
def keylogger(key):
def _print(client):
def listen():
\end{lstlisting}

\begin{lstlisting}

\end{lstlisting}
Chức năng: \\
Tham số: 
\begin{itemize}
\item \lstinline{}
\end{itemize}
Trả về: 
% ------------------------
\begin{lstlisting}
def keylogger(key):
\end{lstlisting}
Chức năng: Dựa vào \lstinline{global} flag (nhận giá trị 0, 1, 2, 4), thực hiện xử lý \lstinline{key} lấy được từ \lstinline{Listener} trong hàm \lstinline{listen()} như sau:\\
+ Thay thế \lstinline{space} thành ký tự khoảng trắng\\
+ Thay thế phím ' thành chuỗi phù hợp \\
+ Lọc bỏ toàn bộ ký tự ' dư thừa \\
+ Nối vào câu \\
Tham số: 
\begin{itemize}
\item \lstinline{key}: Phím được bắt từ quá trình Listener
\end{itemize}
Trả về: Không

\begin{lstlisting}
def _print(client):
\end{lstlisting}
Chức năng: Gửi câu đang được ghi về server\\
Tham số: 
\begin{itemize}
\item \lstinline{client}: socket connection tới server
\end{itemize}
Trả về: Không

\begin{lstlisting}
def listen():
\end{lstlisting}
Chức năng: Tạo thread mới để ghi bàn phím (sử dụng hàm keylogger để xử lý)\\
Trả về: Không

%----
\subsection{Registry}
Sẽ có 2 loại lệnh được gửi cho client:\\
- Liệt kê subkey: SERVER sẽ gửi lệnh \textbf{REGISTRY} cùng với \textbf{LIST} và đường dẫn (ví dụ như là HKEY_CURRENT_USER, HKEY_CURRENT_USER/System), client sẽ trả về danh sách các subkey tương ứng\\
- Cập nhật key: SERVER sẽ gửi lệnh \textbf{REGISTRY} cùng với \textbf{UPDATE} và đường dẫn tới key, giá trị mới của key, và kiểu dữ liệu (kiểu dữ liệu thuộc 1 trong 3 loại: REG_BINARY, REG_DWORD, REG_QWORD)\\

Các hàm cho quá trình này như sau:\\
- Ở server (mail_handler.py):\\

\begin{lstlisting}
def registry_list(ip_address, full_path):
\end{lstlisting}
Chức năng: gửi các câu lệnh để liệt kê subkey với đường dẫn là \lstinline{full_path} đến client, sau đó nhận về danh sách subkey.\\
Tham số: 
\begin{itemize}
\item \lstinline{ip_address}: kiểu tuple(str, int) chứa địa chỉ ip và port của client gửi tới.
\item \lstinline{full_path}: kiểu str, là đường dẫn cần liệt kê subkey.
\end{itemize}

- Ở client (registry_client.py):\\

\begin{lstlisting}
def identify_hkey(value_list):
\end{lstlisting}
Chức năng: Xác định hive từ đường dẫn đã được tách\\
Tham số: 
\begin{itemize}
\item \lstinline{ip_address}: kiểu tuple(str, int) chứa địa chỉ ip và port của client gửi tới.
\item \lstinline{full_path}: kiểu str, là đường dẫn cần liệt kê subkey.
\end{itemize}

\begin{lstlisting}
def registry_list(ip_address, full_path):
\end{lstlisting}
Chức năng: gửi các câu lệnh để liệt kê subkey với đường dẫn là \lstinline{full_path} đến client, sau đó nhận về danh sách subkey.\\
Tham số: 
\begin{itemize}
\item \lstinline{ip_address}: kiểu tuple(str, int) chứa địa chỉ ip và port của client gửi tới.
\item \lstinline{full_path}: kiểu str, là đường dẫn cần liệt kê subkey.
\end{itemize}

\begin{lstlisting}
def registry_list(ip_address, full_path):
\end{lstlisting}
Chức năng: gửi các câu lệnh để liệt kê subkey với đường dẫn là \lstinline{full_path} đến client, sau đó nhận về danh sách subkey.\\
Tham số: 
\begin{itemize}
\item \lstinline{ip_address}: kiểu tuple(str, int) chứa địa chỉ ip và port của client gửi tới.
\item \lstinline{full_path}: kiểu str, là đường dẫn cần liệt kê subkey.
\end{itemize}

% nhiều quá :(