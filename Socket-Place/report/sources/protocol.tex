\section{Kịch bản giao tiếp của chương trình}
\subsection{Giao thức trao đổi giữa client và server: UDP}
\subsection{Cấu trúc của thông điệp:}

\large\textbf{Một thông điệp có tổng cộng \texttt{1024 bytes}.} 
\begin{enumerate}
\item \textbf{Đối với thông điệp từ server có 2 loại sau đây:}
\normalsize
\begin{enumerate}
\item \textit{Thông điệp gửi số lượng thông điệp cần nhận cho data:} \texttt{|padding|len|}

Có dạng \texttt{<1021 bytes 0>xyz} (\texttt{xyz} là ba chữ số).

Gồm 2 trường theo thứ tự đó với các mô tả chi tiết sau đây:

\begin{enumerate}
\item \texttt{padding}: Gồm các null bytes (các bytes \texttt{\textbackslash x00}) chèn vào trước trường \texttt{len} sao cho tổng thông điệp có đủ 1024 bytes.

\item \texttt{len}: kiểu \texttt{str}, thể hiện số lượng thông điệp (tối đa 3 bytes).
\end{enumerate}

\item \textit{Thông điệp gửi từng phân data:} \texttt{|ID|data|hash|}

Có dạng \texttt{xyz<data: 981><hash: 40>} (\texttt{xyz} là ba chữ số).

Gồm 3 trường theo thứ tự đó với các mô tả chi tiết sau đây:

\begin{enumerate}
\item \texttt{ID}: Kiểu \texttt{str} gồm 3 ký tự thể hiện số trong đoạn \texttt{[000; 999]} (3 bytes).

\item \texttt{data}: Kiểu \texttt{byte}, gồm 981 (1024 - 43) ký tự (981 bytes).

\item \texttt{hash}: Kiểu \texttt{str}, gồm 40 ký tự (40 bytes) được tạo khi hash data bằng \texttt{thuật toán SHA1} để kiểm lỗi. \textit{40 bytes này mô tả một số nguyên 20 bytes = 160 bits trong hệ thập lục phân.}
\end{enumerate}


\end{enumerate}
\large
\item \textbf{Đối với thông điệp từ client có 6 loại sau đây:}
\normalsize

\begin{enumerate}
\item \textit{Thông điệp xác nhận số lượng:} \texttt{|specifier|len|padding|}

Có dạng \texttt{ACK{\_}LEN{\_}xyz<1013 bytes 0>} (\texttt{xyz} là ba chữ số).

Gồm 3 trường theo thứ tự đó với các mô tả chi tiết sau đây:

\begin{enumerate}
\item \texttt{specifier}: kiểu \texttt{str}, là chuỗi ký tự \texttt{"ACK\_{LEN}\_"} (8 bytes)
\item \texttt{len}: kiểu \texttt{str}, là số lượng thông điệp đã nhận từ server, gồm 3 ký tự thể hiện số trong đoạn \texttt{[000; 999]} (3 bytes)
\item \texttt{padding}: gồm 1013 (1024 - 8 - 3) bytes \texttt{\textbackslash x00}.
\end{enumerate}

\item \textit{Thông điệp xác nhận phần data:} \texttt{|specifier|id|padding|}

Có dạng \texttt{ACK{\_}xyz<1017 bytes 0>} (\texttt{xyz} là ba chữ số).

Gồm 3 trường theo thứ tự đó với các mô tả chi tiết sau đây:

\begin{enumerate}
\item \texttt{specifier}: kiểu \texttt{str}, là chuỗi ký tự \texttt{"ACK{\_}"} (4 bytes)
\item \texttt{id}: kiểu \texttt{str}, là thứ tự của thông điệp nhận từ server, gồm 3 ký tự thể hiện số trong đoạn [000; 999] (3 bytes)
\item \texttt{padding}: gồm 1017 (1024 - 4 - 3) bytes \texttt{\textbackslash x00}.
\end{enumerate}

\item \textit{Thông điệp yêu cầu thông tin toàn bộ địa điểm:} \texttt{|command|padding|}

Có dạng \texttt{GIV{\_}ALL<1017 bytes 0>}

Gồm 2 trường theo thứ tự đó với các mô tả chi tiết sau đây:

\begin{enumerate}
\item \texttt{command}: kiểu \texttt{str}, là chuỗi ký tự \texttt{"GIV{\_}ALL"} (7 bytes)
\item \texttt{padding}: gồm 1017 (1024 - 7) bytes \texttt{\textbackslash x00}.

\end{enumerate}

\item \textit{Thông điệp yêu cầu thông tin chi tiết 1 địa điểm:} \texttt{|command|id|padding|}

Có dạng \texttt{GIV{\_}DETAIL{\_}<id: x bytes><1013 - x bytes 0>} 

Gồm 3 trường theo thứ tự đó với các mô tả chi tiết sau đây:

\begin{enumerate}
\item \texttt{command}: kiểu \texttt{str}, là chuỗi ký tự \texttt{"GIV{\_}DETAIL{\_}"} (11 bytes)
\item \texttt{id}: kiểu \texttt{str}, thể hiện id của địa điểm cần lấy.
\item \texttt{padding}: Gồm các bytes \texttt{\textbackslash x00} điền cho đủ 1024 bytes.
\end{enumerate}

\item \textit{Thông điệp yêu cầu ảnh đại diện 1 địa điểm:} \texttt{|command|id|padding |}

Có dạng \texttt{GIV{\_}AVT{\_}<id: x bytes><1016 - x bytes 0>} 

Gồm 3 trường theo thứ tự đó với các mô tả chi tiết sau đây:

\begin{enumerate}
\item \texttt{command}: kiểu \texttt{str}, là chuỗi ký tự \texttt{"GIV{\_}AVT{\_}"} (8 bytes)
\item \texttt{id}: kiểu \texttt{str}, thể hiện id của địa điểm cần lấy.
\item \texttt{padding}: Gồm các bytes \texttt{\textbackslash x00} điền cho đủ 1024 bytes.
\end{enumerate}

\item \textit{Thông điệp yêu cầu ảnh tại 1 địa điểm:} \texttt{|command|position|separator|id|padding|}

Có dạng \texttt{GIV{\_}IMG{\_}xyz{\_}<id: t bytes><1012 - t bytes 0>} (\texttt{xyz} là ba chữ số).

Gồm 5 trường theo thứ tự đó với các mô tả chi tiết sau đây:

\begin{enumerate}
\item \texttt{command}: kiểu \texttt{str}, là chuỗi ký tự \texttt{"GIV{\_}IMG{\_}"} (8 bytes)
\item \texttt{position}: kiểu \texttt{str}, là thứ tự trong danh sách ảnh cần lấy, gồm 3 ký tự thể hiện số trong đoạn \texttt{[000; 999]} (3 bytes)
\item \texttt{separator}: kiểu char, là ký tự '{\_}'
\item \texttt{id}: kiểu \texttt{str}, thể hiện id của địa điểm cần lấy.
\item \texttt{padding}: Gồm các bytes \texttt{\textbackslash x00} điền cho đủ 1024 bytes.
\end{enumerate}
\end{enumerate}
\end{enumerate}
\subsection{Cách tổ chức cơ sở dữ liệu:}
\subsection{Quá trình truyền từ client lên server:}
